%---------------------------Radius Ratio----------------------------------
\section{Radius Ratio\label{s:tet-radius-ratio}}

This metric is commonly known as the radius ratio since it is the
normalized ratio of the radius of the inscribed sphere to the radius
of the circumsphere. Note that it is equal to the tetrahedral aspect
$\beta$ for positively-oriented tetrahedra.  

The radius ratio is the quotient of these two radii normalized by $\frac{1}{3}$ so
that an equilateral tetrahedron has quality of 1:
\begin{eqnarray*}
q & = & \frac{R}{3 r} \nonumber \\
  & = & \frac { \left| 
   \normvec{L_3}^2 \left( \vec L_2 \times \vec L_0 \right) + 
   \normvec{L_2}^2 \left( \vec L_3 \times \vec L_0 \right) + 
   \normvec{L_0}^2 \left( \vec L_3 \times \vec L_2 \right)
   \right| A}{108 V^2}.
\end{eqnarray*}

Note that if $|V| < DBL\_MIN$, we set $q = DBL\_MAX$.

\tetmetrictable{radius ratio}%
{$1$}%                  Dimension
{$[1,3]$}%              Acceptable range
{$[1,DBL\_MAX]$}%       Normal range
{$[1,DBL\_MAX]$}%       Full range
{$1$}%                  Equilateral tet
{\cite{par:93}}%        Citation
{v\_tet\_radius\_ratio}% Verdict function name


